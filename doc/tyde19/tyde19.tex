\documentclass[sigplan,screen]{acmart}

\usepackage{booktabs} % For formal tables

% packages used by me
\usepackage{agda}
\usepackage{catchfilebetweentags}
\usepackage{balance}

%%% The following is specific to TyDe '19 and the paper
%%% 'Generic Level Polymorphic N-ary Functions'
%%% by Guillaume Allais.
%%%
\setcopyright{acmlicensed}
\acmPrice{15.00}
\acmDOI{10.1145/3331554.3342604}
\acmYear{2019}
\copyrightyear{2019}
\acmISBN{978-1-4503-6815-5/19/08}
\acmConference[TyDe '19]{Proceedings of the 4th ACM SIGPLAN International Workshop on Type-Driven Development}{August 18, 2019}{Berlin, Germany}
\acmBooktitle{Proceedings of the 4th ACM SIGPLAN International Workshop on Type-Driven Development (TyDe '19), August 18, 2019, Berlin, Germany}


%%%%%%%%%% AGDA ALIASES

\newcommand{\APT}{\AgdaPrimitiveType}
\newcommand{\AK}{\AgdaKeyword}
\newcommand{\AM}{\AgdaModule}
\newcommand{\AS}{\AgdaSymbol}
\newcommand{\AStr}{\AgdaString}
\newcommand{\AN}{\AgdaNumber}
\newcommand{\AD}{\AgdaDatatype}
\newcommand{\AF}{\AgdaFunction}
\newcommand{\AR}{\AgdaRecord}
\newcommand{\ARF}{\AgdaField}
\newcommand{\AB}{\AgdaBound}
\newcommand{\AIC}{\AgdaInductiveConstructor}
\newcommand{\AC}{\AgdaComment}

\newcommand{\Set}{\mathbf{Set}}


\usepackage[utf8]{inputenc}
\usepackage{newunicodechar}
\usepackage{amssymb}
\usepackage{txfonts}

% Misc symbols
\newunicodechar{′}{\ensuremath{\prime}}
\newunicodechar{−}{\ensuremath{-}}
\newunicodechar{─}{\ensuremath{-}}
\newunicodechar{◆}{\ensuremath{\Diamondblack}}
\newunicodechar{⧫}{\ensuremath{\blacklozenge}}
\newunicodechar{∷}{\ensuremath{::}}
\newunicodechar{∙}{\ensuremath{\bullet}}
\newunicodechar{□}{\ensuremath{\Box}}
\newunicodechar{∎}{\ensuremath{\blacksquare}}
\newunicodechar{⋆}{\ensuremath{\star}}

% indices
\newunicodechar{₀}{\ensuremath{_0}}
\newunicodechar{₁}{\ensuremath{_1}}
\newunicodechar{₂}{\ensuremath{_2}}
\newunicodechar{₃}{\ensuremath{_3}}

\newunicodechar{ₙ}{\ensuremath{_n}}
\newunicodechar{ᵣ}{\ensuremath{_r}}
\newunicodechar{ₛ}{\ensuremath{_s}}

% exponents
\newunicodechar{⁺}{\ensuremath{^+}}
\newunicodechar{⁻}{\ensuremath{^-}}

\newunicodechar{²}{\ensuremath{^2}}

\newunicodechar{ⁱ}{\ensuremath{^i}}
\newunicodechar{ˡ}{\ensuremath{^l}}
\newunicodechar{ʳ}{\ensuremath{^r}}

\newunicodechar{ᴬ}{\ensuremath{^A}}
\newunicodechar{ᴮ}{\ensuremath{^B}}
\newunicodechar{ᴵ}{\ensuremath{^I}}
\newunicodechar{ᴿ}{\ensuremath{^R}}
\newunicodechar{ᵀ}{\ensuremath{^T}}
\newunicodechar{ᵁ}{\ensuremath{^U}}
\newunicodechar{ⱽ}{\ensuremath{^V}}

% Dots
\newunicodechar{⋯}{\ensuremath{\cdots}}

% Equality symbols
\newunicodechar{≡}{\ensuremath{\equiv}}
\newunicodechar{≢}{\ensuremath{\not\equiv}}
\newunicodechar{≟}{\mbox{\tiny\ensuremath{\stackrel{?}{=}}}}
\newunicodechar{≈}{\ensuremath{\approx}}

% Ordering symbols
\newunicodechar{≤}{\ensuremath{\le}}
\newunicodechar{≥}{\ensuremath{\ge}}

% Arrows
\newunicodechar{→}{\ensuremath{\rightarrow}}
\newunicodechar{←}{\ensuremath{\leftarrow}}
\newunicodechar{⇒}{\ensuremath{\Rightarrow}}
\newunicodechar{⇉}{\ensuremath{\rightrightarrows}}

% Mathematical symbols
\newunicodechar{∂}{\ensuremath{\partial}}
\newunicodechar{∋}{\ensuremath{\ni}}
\newunicodechar{∞}{\ensuremath{\infty}}
\newunicodechar{∀}{\ensuremath{\forall}}
\newunicodechar{∃}{\ensuremath{\exists}}
\newunicodechar{⊢}{\ensuremath{\vdash}}
\newunicodechar{⟨}{\ensuremath{\langle}}
\newunicodechar{⟩}{\ensuremath{\rangle}}
\newunicodechar{⊤}{\ensuremath{\top}}
\newunicodechar{∘}{\ensuremath{\circ}}
\newunicodechar{⊎}{\ensuremath{\uplus}}
\newunicodechar{×}{\ensuremath{\times}}
\newunicodechar{ℕ}{\ensuremath{\mathbb{N}}}
\newunicodechar{⟦}{\ensuremath{\llbracket}}
\newunicodechar{⟧}{\ensuremath{\rrbracket}}
\newunicodechar{∈}{\ensuremath{\in}}
\newunicodechar{↑}{\ensuremath{\uparrow}}
\newunicodechar{¬}{\ensuremath{\neg}}
\newunicodechar{⊥}{\ensuremath{\bot}}
\newunicodechar{↝}{\ensuremath{\leadsto}}
\newunicodechar{↶}{\ensuremath{\curvearrowleft}}
\newunicodechar{↺}{\ensuremath{\circlearrowleft}}
\newunicodechar{⊔}{\ensuremath{\sqcup}}


% Greek uppercase
\newunicodechar{Δ}{\ensuremath{\Delta}}
\newunicodechar{Γ}{\ensuremath{\Gamma}}
\newunicodechar{Σ}{\ensuremath{\Sigma}}
\newunicodechar{Θ}{\ensuremath{\Theta}}
\newunicodechar{Ω}{\ensuremath{\Omega}}

% Greek lowercase
\newunicodechar{α}{\ensuremath{\alpha}}
\newunicodechar{β}{\ensuremath{\beta}}
\newunicodechar{δ}{\ensuremath{\delta}}
\newunicodechar{ε}{\ensuremath{\varepsilon}}
\newunicodechar{φ}{\ensuremath{\phi}}
\newunicodechar{γ}{\ensuremath{\gamma}}
\newunicodechar{ι}{\ensuremath{\iota}}
\newunicodechar{κ}{\ensuremath{\kappa}}
\newunicodechar{λ}{\ensuremath{\lambda}}
\newunicodechar{μ}{\ensuremath{\mu}}
\newunicodechar{ψ}{\ensuremath{\psi}}
\newunicodechar{η}{\ensuremath{\eta}}
\newunicodechar{ρ}{\ensuremath{\rho}}
\newunicodechar{σ}{\ensuremath{\sigma}}
\newunicodechar{τ}{\ensuremath{\tau}}
\newunicodechar{ξ}{\ensuremath{\xi}}
\newunicodechar{ζ}{\ensuremath{\zeta}}
\newunicodechar{Π}{\ensuremath{\Pi}}


% mathcal
\newunicodechar{𝓒}{\ensuremath{\mathcal{C}}}
\newunicodechar{𝓔}{\ensuremath{\mathcal{E}}}
\newunicodechar{𝓕}{\ensuremath{\mathcal{F}}}
\newunicodechar{𝓡}{\ensuremath{\mathcal{R}}}
\newunicodechar{𝓢}{\ensuremath{\mathcal{S}}}
\newunicodechar{𝓣}{\ensuremath{\mathcal{T}}}
\newunicodechar{𝓥}{\ensuremath{\mathcal{V}}}
\newunicodechar{𝓦}{\ensuremath{\mathcal{W}}}


\begin{CCSXML}
<ccs2012>
<concept>
<concept_id>10003752.10010124.10010125.10010130</concept_id>
<concept_desc>Theory of computation~Type structures</concept_desc>
<concept_significance>500</concept_significance>
</concept>
<concept>
<concept_id>10011007.10011006.10011050.10011017</concept_id>
<concept_desc>Software and its engineering~Domain specific languages</concept_desc>
<concept_significance>500</concept_significance>
</concept>
<concept>
<concept_id>10011007.10011006.10011072</concept_id>
<concept_desc>Software and its engineering~Software libraries and repositories</concept_desc>
<concept_significance>500</concept_significance>
</concept>
</ccs2012>
\end{CCSXML}

\ccsdesc[500]{Theory of computation~Type structures}
\ccsdesc[500]{Software and its engineering~Domain specific languages}
\ccsdesc[500]{Software and its engineering~Software libraries and repositories}

\keywords{Dependent types, Arity-generic programming, Universe polymorphism, Agda}

\begin{document}
\title{}
%\titlenote{Produces the permission block, and
%  copyright information}
%\subtitle{Extended Abstract}
%\subtitlenote{The full version of the author's guide is available as
%  \texttt{acmart.pdf} document}

\title{Generic Level Polymorphic N-ary Functions}

\author{Guillaume Allais}
\affiliation{%
  \institution{University of Strathclyde}
  \city{Glasgow}
  \country{UK}
}
\email{guillaume.allais@strath.ac.uk}

% The default list of authors is too long for headers.
% \renewcommand{\shortauthors}{B. Trovato et al.}

\begin{abstract}
Agda's standard library struggles in various places with n-ary functions and
relations. It introduces congruence and substitution operators for functions
of arities one and two, and provides users with convenient combinators for
manipulating indexed families of arity exactly one.

After a careful analysis of the kinds of problems the unifier can easily solve,
we design a unifier-friendly representation of n-ary functions. This allows us
to write generic programs acting on n-ary functions which automatically reconstruct
the representation of their inputs' types by unification. In particular, we can
define fully level polymorphic n-ary versions of congruence, substitution and the
combinators for indexed families, all requiring minimal user input.
\end{abstract}

\maketitle


\section{N-ary combinators... for N up to 2}

\ExecuteMetaData[StateOfTheArt.tex]{equality}

\ExecuteMetaData[StateOfTheArt.tex]{cong}
\ExecuteMetaData[StateOfTheArt.tex]{subst}

\ExecuteMetaData[StateOfTheArt.tex]{cong2}
\ExecuteMetaData[StateOfTheArt.tex]{subst2}


\section{Working with Indexed Families}


\ExecuteMetaData[StateOfTheArt.tex]{list}
\ExecuteMetaData[StateOfTheArt.tex]{all}
\ExecuteMetaData[StateOfTheArt.tex]{exists}
\ExecuteMetaData[StateOfTheArt.tex]{satisfiable}

\ExecuteMetaData[StateOfTheArt.tex]{iuniversal}
\ExecuteMetaData[StateOfTheArt.tex]{implies}
\ExecuteMetaData[StateOfTheArt.tex]{ap}

\ExecuteMetaData[StateOfTheArt.tex]{negation}
\ExecuteMetaData[StateOfTheArt.tex]{any}
\ExecuteMetaData[StateOfTheArt.tex]{none}

\ExecuteMetaData[StateOfTheArt.tex]{universal}
\ExecuteMetaData[StateOfTheArt.tex]{replicate}

\section{Working With Multiple Indices}

We started by showing both the type and the implementation of each of our
examples. Although convenient at first to build an understanding of which
arguments are explicit and which ones are implicit, we are in the end only
interested about the way combinators let us write types. From now on, we
focus on the types and only the types of our examples.

The combinators presented earlier which are all available in the standard
library work well for unary predicates. Unfortunately they do not scale
beyond that. Meaning that if we are manipulating binary relations for
instance we have to partially apply them before we can use our combinators.
This leads to cluttered types which are not much better than their fully
explicit counterparts. Assuming that \AF{\_≤\_} is the usual order on natural
numbers and \AF{\_≥\_} is its converse, to state that \AF{\_≤\_} is antisymmetric
we have to write:

\ExecuteMetaData[StateOfTheArt.tex]{brokenantisym}

When ideally we would have written:

\ExecuteMetaData[Examples.tex]{antisym}


\appendix
\section{Agda-Specific Features}\label{appendix:agda}

We provide here a description of some of the more esoteric Agda features
used in this paper. Readers interested in a more thorough introduction to
the language may consider reading Ulf Norell's lecture
notes~\cite{DBLP:conf/afp/Norell08}.

\subsection{Syntax Highlighting}

The colours used in this paper all have a meaning: keywords are highlighted
in \AK{orange}; \AF{blue} denotes function and type definitions; \AIC{green}
marks constructors; \ARF{pink} is associated to record fields and the
corresponding projections.


\subsection{Implicit Variable Generalisation}\label{appendix:agda:variable}

Agda supports the implicit generalisation of variables appearing in type
signatures. Every time a seemingly unbound variable is used, the reader
can safely assume that it was automatically introduced by Agda using a
prenex implicit quantifier. Both Haskell and OCaml behave similarly with
respect to unbound type variables.

In the type of the following definition for instance, \AB{A} and \AB{B} are
two \AF{Set}s of respective universe levels \AB{a} and \AB{b} (see
Appendix~\ref{appendix:agda:level}) and \AB{x} and \AB{y} are two values of
type \AB{A}. All of these variables have been introduced using this implicit
generalisation mechanism.

\ExecuteMetaData[StateOfTheArt.tex]{cong}

If we had not relied on the implicit generalisation mechanism, we would have
needed to write the following verbose type declaration.

\ExecuteMetaData[Appendix.tex]{congtype}

This mechanism can also be used when defining an inductive family.
In Section~\ref{def:all}, we introduced the predicate lifting \AD{All}
in the following mannger. The careful reader will have noticed a
number of unbound names: \AB{a}, \AB{A}, \AB{p} in the declaration
of the type constructor and \AB{x} and \AB{xs} in the declaration of
the data constructor \AIC{\_::\_}.

\ExecuteMetaData[StateOfTheArt.tex]{all}

This definition corresponds internally to the following expanded
version (modulo the order in which the variables have been generalised
over).

\ExecuteMetaData[Appendix.tex]{all}


\subsection{Meaning of Underscore}

Underscores have different meanings in different contexts. They can either stand
for argument positions when defining identifiers, trivial values Agda should be
able to reconstruct, or discarded values.

\subsubsection{Argument Position in a Mixfix Identifier}

Users can define arbitrary mixfix identifiers as names for both functions and
constructors. Mixfix identifiers are a generalisation of infix identifiers
which turns any alternating list of name parts and argument positions into a
valid identifier~\cite{DBLP:conf/ifl/DanielssonN08}. These argument positions
are denoted using an underscore. For instance \AF{∀[\_]} is a unary operator,
(\AIC{\_::\_}) corresponds to a binary infix identifier and (\AF{\_\%=\_⊢\_}) is a
ternary operator.

\subsubsection{Trivial Value}

Programmers can leave out trivial parts of a definition by using an underscore
instead of spelling out the tedious details. This will be accepted by Agda as
long as it is able to reconstruct the missing value by unification. We discuss
these use cases further in Section~\ref{sec:unificationtest}.

\subsubsection{Ignored Binder}

An underscore used in place of an identifier in a binder means that the binding
should be discarded. For instance {(λ \_ → a)} defines a constant function.
Toplevel bindings can similarly be discarded which is a convenient way of
writing unit tests (in type theory programs can be run at typechecking time)
without polluting the namespace. The following unnamed definition checks for
instance the result of applying addition defined on natural numbers to
\AN{2} and \AN{3}.

\ExecuteMetaData[Appendix.tex]{unittest}

\subsection{Universe levels}\label{appendix:agda:level}

Agda avoids Russell-style paradoxes by introducing a tower of universes
\AF{Set₀} (usually written \AF{Set}), \AF{Set₁}, \AF{Set₂}, etc. Each
\AF{Setₙ} does not itself have type \AF{Setₙ} but rather \AF{Setₙ₊₁} thus
preventing circularity.

We can form {\bf function types} from a domain type in \AF{Setₘ} to a codomain
type in \AF{Setₙ}. Such a function type lives at the level corresponding
to the maxium of \AB{m} and \AB{n}. This maximum is denoted {(\AB{m} \AF{⊔} \AB{n})}.

An {\bf inductive} type or a {\bf record} type storing values of type \AF{Setₙ}
needs to be defined at universe level \AB{n} or higher. We can combine multiple
constraints of this form by using the maximum operator. The respective definitions
of propositional equality in Section~\ref{sec:nary2} and dependent pairs in
Section~\ref{def:sigma} are examples of such data and record types.

Without support for a mechanism to define {\bf level polymorphic functions},
we would need to duplicate a lot of code. Luckily Agda has a primitive notion
of universe levels called \AD{Level}. We can write level polymorphic code by
quantifying over such a \AD{Level} \AB{l} and form the universe at level
\AB{l} by writing (\AF{Set} \AB{l}). The prototypical example of such a level
polymorphic function is the identity function \AF{id} defined as follows.

\ExecuteMetaData[Appendix.tex]{identity}


\section*{Acknowledgements}

We would like to thank the reviewers for their helpful comments and their
suggestions to discuss parametricity as a derivation principle, and to add
an appendix to make the paper accessible to a wider audience.

The research leading to these results has received funding from EPSRC
grant EP/M016951/1.

\bibliographystyle{ACM-Reference-Format}
\bibliography{nary}

\end{document}
