\section{Agda-Specific Features}\label{appendix:agda}

We provide here a description of some of the more esoteric Agda
features used in this paper. Readers interested in a more thorough
introduction to the language may consider reading Ulf Norell's lecture
notes~\cite{DBLP:conf/afp/Norell08}.

\subsection{Syntax Highlighting in the Text}

The colours used in this paper all have a meaning: keywords are highlighted
in \AK{orange}; \AF{blue} denotes function and type definitions; \AIC{green}
marks constructors; \ARF{pink} is associated to record fields and
corresponding projections.

\subsection{Universe Levels}\label{appendix:agda:level}
\balance

Agda avoids Russell-style paradoxes by introducing a tower of universes
\AF{Set₀} (usually written \AF{Set}), \AF{Set₁}, \AF{Set₂}, etc. Each
\AF{Setₙ} does not itself have type \AF{Setₙ} but rather \AF{Setₙ₊₁} thus
preventing circularity.

We can form {\bf function types} from a domain type in \AF{Setₘ} to a codomain
type in \AF{Setₙ}. Such a function type lives at the level corresponding
to the maximum of \AB{m} and \AB{n}. This maximum is denoted {(\AB{m} \AF{⊔} \AB{n})}.

An {\bf inductive} type or a {\bf record} type storing values of type \AF{Setₙ}
needs to be defined at universe level \AB{n} or higher. We can combine multiple
constraints of this form by using the maximum operator. The respective definitions
of propositional equality in Section~\ref{sec:nary2} and dependent pairs in
Section~\ref{def:sigma} are examples of such data and record types.

Without support for a mechanism to define {\bf level polymorphic functions},
we would need to duplicate a lot of code. Luckily Agda has a primitive notion
of universe levels called \AD{Level}. We can write level polymorphic code by
quantifying over such a level \AB{l} and form the universe at level
\AB{l} by writing (\AF{Set} \AB{l}). The prototypical example of such a level
polymorphic function is the identity function \AF{id} defined as follows.

\ExecuteMetaData[Appendix.tex]{identity}

\subsection{Meaning of Underscore}

Underscores have different meanings in different contexts. They can either stand
for argument positions when defining identifiers, trivial values Agda should be
able to reconstruct, or discarded values.

\subsubsection{Argument Position in a Mixfix Identifier}

Users can define arbitrary mixfix identifiers as names for both functions and
constructors. Mixfix identifiers are a generalisation of infix identifiers
which turns any alternating list of name parts and argument positions into a
valid identifier~\cite{DBLP:conf/ifl/DanielssonN08}. These argument positions
are denoted using an underscore. For instance \AF{∀[\_]} is a unary operator,
(\AIC{\_::\_}) corresponds to a binary infix identifier and (\AF{\_\%=\_⊢\_}) is a
ternary operator.

\subsubsection{Trivial Value}

Programmers can leave out trivial parts of a definition by using an underscore
instead of spelling out the tedious details. This will be accepted by Agda as
long as it is able to reconstruct the missing value by unification. We discuss
these use cases further in Section~\ref{sec:unificationtest}.

\subsubsection{Ignored Binder}

An underscore used in place of an identifier in a binder means that the binding
should be discarded. For instance {(λ \_ → a)} defines a constant function.
Toplevel bindings can similarly be discarded which is a convenient way of
writing unit tests (in type theory programs can be run at typechecking time)
without polluting the namespace. The following unnamed definition checks for
instance the result of applying addition defined on natural numbers to
\AN{2} and \AN{3}.

\ExecuteMetaData[Appendix.tex]{unittest}

\subsection{Implicit Variable Generalisation}\label{appendix:agda:variable}

Agda supports the implicit generalisation of variables appearing in type
signatures. Every time a seemingly unbound variable is used, the reader can
safely assume that it was declared by us as being a \AK{variable} Agda should
automatically introduce. These variables are bound using an implicit
prenex universal quantifier. Haskell, OCaml, and Idris behave similarly with
respect to unbound type variables.

In the type of the following definition for instance, \AB{A} and \AB{B} are
two \AF{Set}s of respective universe levels \AB{a} and \AB{b} (see
Appendix~\ref{appendix:agda:level}) and \AB{x} and \AB{y} are two values of
type \AB{A}. All of these variables have been introduced using this implicit
generalisation mechanism.

\ExecuteMetaData[StateOfTheArt.tex]{cong}

If we had not relied on the implicit generalisation mechanism, we would have
needed to write the following verbose type declaration.

\ExecuteMetaData[Appendix.tex]{congtype}

This mechanism can also be used when defining an inductive family.
In Section~\ref{def:all}, we introduced the predicate lifting \AD{All}
in the following manner. The careful reader will have noticed a
number of unbound names: \AB{a}, \AB{A}, \AB{p} in the declaration
of the type constructor and \AB{x} and \AB{xs} in the declaration of
the data constructor \AIC{\_::\_}.

\ExecuteMetaData[StateOfTheArt.tex]{all}

This definition corresponds internally to the following expanded
version (modulo the order in which the variables have been generalised
over).

\ExecuteMetaData[Appendix.tex]{all}
