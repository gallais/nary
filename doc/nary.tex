For user convenience, Agda's standard library has accumulated a set of
equality-manipulating combinators of varying arities (Section~\ref{sec:nary2})
as well as a type-level compositional DSL to write clean types involving
indexed families of arity exactly one (Section~\ref{sec:unarycombinators}).
None of these solutions scale well.

\section{N-ary combinators... for N up to 2}\label{sec:nary2}

Agda's standard library relies on propositional equality defined as
a level polymorphic inductive family with one constructor \AIC{refl}.

\ExecuteMetaData[StateOfTheArt.tex]{equality}

As one would expect from a notion of equality, it is congruent (i.e.
for any function equal inputs yield equal outputs) and substitutive
(i.e. equals behave the same with respect to predicates). Concretely
this means we can write the two following functions by dependent
pattern-matching on the equality proof:

\ExecuteMetaData[StateOfTheArt.tex]{cong}
\ExecuteMetaData[StateOfTheArt.tex]{subst}

However we quickly realise that it is convenient to be able to use
congruence for functions which take more than one argument and
substitution for at least binary relations. The standard library
provides binary versions of both of these functions:

\ExecuteMetaData[StateOfTheArt.tex]{cong2}
\ExecuteMetaData[StateOfTheArt.tex]{subst2}

If we want to go beyond arity two however we are left to either define
our own ternary, quaternary, etc. versions of \AF{cong} and \AF{subst}
or awkwardly chaining the ones with a lower arity to slowly massage
the expression at hand into the shape we want. Both of these solutions
are unsatisfactory.

\paragraph{Wish} Ideally, we would like to define once and for all two
functions \AF{congₙ} and \AF{substₙ} of respective types (pseudocode):

\medskip
\noindent\begin{tabular}{@{}l@{~}l}
  \AF{congₙ} : & (\AB{f} : \AB{A₁} → ⋯ → \AB{Aₙ} → \AB{B}) →\\
               & \AB{a₁} \AD{≡} \AB{b₁} → ⋯ \AB{aₙ} \AD{≡} \AB{bₙ} → \\
               & \AB{f} \AB{a₁} ⋯ \AB{aₙ} \AD{≡} \AB{f} \AB{b₁} ⋯ \AB{bₙ}
\end{tabular}
\medskip

\noindent\begin{tabular}{@{}l@{~}l}
  \AF{substₙ} : & (\AB{R} : \AB{A₁} → ⋯ → \AB{Aₙ} → \AF{Set} \AB{r}) →\\
                & \AB{a₁} \AD{≡} \AB{b₁} → ⋯ \AB{aₙ} \AD{≡} \AB{bₙ} → \\
                & \AB{R} \AB{a₁} ⋯ \AB{aₙ} → \AB{R} \AB{b₁} ⋯ \AB{bₙ}
\end{tabular}

\section{Working with Indexed Families}


\ExecuteMetaData[StateOfTheArt.tex]{list}
\ExecuteMetaData[StateOfTheArt.tex]{all}
\ExecuteMetaData[StateOfTheArt.tex]{exists}
\ExecuteMetaData[StateOfTheArt.tex]{satisfiable}

\ExecuteMetaData[StateOfTheArt.tex]{iuniversal}
\ExecuteMetaData[StateOfTheArt.tex]{implies}
\ExecuteMetaData[StateOfTheArt.tex]{ap}

\ExecuteMetaData[StateOfTheArt.tex]{negation}
\ExecuteMetaData[StateOfTheArt.tex]{any}
\ExecuteMetaData[StateOfTheArt.tex]{none}

\ExecuteMetaData[StateOfTheArt.tex]{universal}
\ExecuteMetaData[StateOfTheArt.tex]{replicate}

\section{Working With Multiple Indices}

We started by showing both the type and the implementation of each of our
examples. Although convenient at first to build an understanding of which
arguments are explicit and which ones are implicit, we are in the end only
interested about the way combinators let us write types. From now on, we
focus on the types and only the types of our examples.

The combinators presented earlier which are all available in the standard
library work well for unary predicates. Unfortunately they do not scale
beyond that. Meaning that if we are manipulating binary relations for
instance we have to partially apply them before we can use our combinators.
This leads to cluttered types which are not much better than their fully
explicit counterparts. Assuming that \AF{\_≤\_} is the usual order on natural
numbers and \AF{\_≥\_} is its converse, to state that \AF{\_≤\_} is antisymmetric
we have to write:

\ExecuteMetaData[StateOfTheArt.tex]{brokenantisym}

When ideally we would have written:

\ExecuteMetaData[Examples.tex]{antisym}
