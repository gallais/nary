\documentclass[compress,9pt]{beamer}

\usepackage[sci]{strathclyde}
\makeatletter
\let\code\@undefined
\usepackage{agda}
\usepackage{catchfilebetweentags}
\usepackage[T1]{fontenc}

\patchcmd{\beamer@sectionintoc}
  {\vfill}
  {\vskip\itemsep}
  {}
  {}


\usepackage[utf8]{inputenc}
\usepackage{newunicodechar}
\usepackage{amssymb}
\usepackage{txfonts}

% Misc symbols
\newunicodechar{′}{\ensuremath{\prime}}
\newunicodechar{−}{\ensuremath{-}}
\newunicodechar{─}{\ensuremath{-}}
\newunicodechar{◆}{\ensuremath{\Diamondblack}}
\newunicodechar{⧫}{\ensuremath{\blacklozenge}}
\newunicodechar{∷}{\ensuremath{::}}
\newunicodechar{∙}{\ensuremath{\bullet}}
\newunicodechar{□}{\ensuremath{\Box}}
\newunicodechar{∎}{\ensuremath{\blacksquare}}
\newunicodechar{⋆}{\ensuremath{\star}}

% indices
\newunicodechar{₀}{\ensuremath{_0}}
\newunicodechar{₁}{\ensuremath{_1}}
\newunicodechar{₂}{\ensuremath{_2}}
\newunicodechar{₃}{\ensuremath{_3}}

\newunicodechar{ₙ}{\ensuremath{_n}}
\newunicodechar{ᵣ}{\ensuremath{_r}}
\newunicodechar{ₛ}{\ensuremath{_s}}

% exponents
\newunicodechar{⁺}{\ensuremath{^+}}
\newunicodechar{⁻}{\ensuremath{^-}}

\newunicodechar{²}{\ensuremath{^2}}

\newunicodechar{ⁱ}{\ensuremath{^i}}
\newunicodechar{ˡ}{\ensuremath{^l}}
\newunicodechar{ʳ}{\ensuremath{^r}}

\newunicodechar{ᴬ}{\ensuremath{^A}}
\newunicodechar{ᴮ}{\ensuremath{^B}}
\newunicodechar{ᴵ}{\ensuremath{^I}}
\newunicodechar{ᴿ}{\ensuremath{^R}}
\newunicodechar{ᵀ}{\ensuremath{^T}}
\newunicodechar{ᵁ}{\ensuremath{^U}}
\newunicodechar{ⱽ}{\ensuremath{^V}}

% Dots
\newunicodechar{⋯}{\ensuremath{\cdots}}

% Equality symbols
\newunicodechar{≡}{\ensuremath{\equiv}}
\newunicodechar{≢}{\ensuremath{\not\equiv}}
\newunicodechar{≟}{\mbox{\tiny\ensuremath{\stackrel{?}{=}}}}
\newunicodechar{≈}{\ensuremath{\approx}}

% Ordering symbols
\newunicodechar{≤}{\ensuremath{\le}}
\newunicodechar{≥}{\ensuremath{\ge}}

% Arrows
\newunicodechar{→}{\ensuremath{\rightarrow}}
\newunicodechar{←}{\ensuremath{\leftarrow}}
\newunicodechar{⇒}{\ensuremath{\Rightarrow}}
\newunicodechar{⇉}{\ensuremath{\rightrightarrows}}

% Mathematical symbols
\newunicodechar{∂}{\ensuremath{\partial}}
\newunicodechar{∋}{\ensuremath{\ni}}
\newunicodechar{∞}{\ensuremath{\infty}}
\newunicodechar{∀}{\ensuremath{\forall}}
\newunicodechar{∃}{\ensuremath{\exists}}
\newunicodechar{⊢}{\ensuremath{\vdash}}
\newunicodechar{⟨}{\ensuremath{\langle}}
\newunicodechar{⟩}{\ensuremath{\rangle}}
\newunicodechar{⊤}{\ensuremath{\top}}
\newunicodechar{∘}{\ensuremath{\circ}}
\newunicodechar{⊎}{\ensuremath{\uplus}}
\newunicodechar{×}{\ensuremath{\times}}
\newunicodechar{ℕ}{\ensuremath{\mathbb{N}}}
\newunicodechar{⟦}{\ensuremath{\llbracket}}
\newunicodechar{⟧}{\ensuremath{\rrbracket}}
\newunicodechar{∈}{\ensuremath{\in}}
\newunicodechar{↑}{\ensuremath{\uparrow}}
\newunicodechar{¬}{\ensuremath{\neg}}
\newunicodechar{⊥}{\ensuremath{\bot}}
\newunicodechar{↝}{\ensuremath{\leadsto}}
\newunicodechar{↶}{\ensuremath{\curvearrowleft}}
\newunicodechar{↺}{\ensuremath{\circlearrowleft}}
\newunicodechar{⊔}{\ensuremath{\sqcup}}


% Greek uppercase
\newunicodechar{Δ}{\ensuremath{\Delta}}
\newunicodechar{Γ}{\ensuremath{\Gamma}}
\newunicodechar{Σ}{\ensuremath{\Sigma}}
\newunicodechar{Θ}{\ensuremath{\Theta}}
\newunicodechar{Ω}{\ensuremath{\Omega}}

% Greek lowercase
\newunicodechar{α}{\ensuremath{\alpha}}
\newunicodechar{β}{\ensuremath{\beta}}
\newunicodechar{δ}{\ensuremath{\delta}}
\newunicodechar{ε}{\ensuremath{\varepsilon}}
\newunicodechar{φ}{\ensuremath{\phi}}
\newunicodechar{γ}{\ensuremath{\gamma}}
\newunicodechar{ι}{\ensuremath{\iota}}
\newunicodechar{κ}{\ensuremath{\kappa}}
\newunicodechar{λ}{\ensuremath{\lambda}}
\newunicodechar{μ}{\ensuremath{\mu}}
\newunicodechar{ψ}{\ensuremath{\psi}}
\newunicodechar{η}{\ensuremath{\eta}}
\newunicodechar{ρ}{\ensuremath{\rho}}
\newunicodechar{σ}{\ensuremath{\sigma}}
\newunicodechar{τ}{\ensuremath{\tau}}
\newunicodechar{ξ}{\ensuremath{\xi}}
\newunicodechar{ζ}{\ensuremath{\zeta}}
\newunicodechar{Π}{\ensuremath{\Pi}}


% mathcal
\newunicodechar{𝓒}{\ensuremath{\mathcal{C}}}
\newunicodechar{𝓔}{\ensuremath{\mathcal{E}}}
\newunicodechar{𝓕}{\ensuremath{\mathcal{F}}}
\newunicodechar{𝓡}{\ensuremath{\mathcal{R}}}
\newunicodechar{𝓢}{\ensuremath{\mathcal{S}}}
\newunicodechar{𝓣}{\ensuremath{\mathcal{T}}}
\newunicodechar{𝓥}{\ensuremath{\mathcal{V}}}
\newunicodechar{𝓦}{\ensuremath{\mathcal{W}}}

%%%%%%%%%% AGDA ALIASES

\newcommand{\APT}{\AgdaPrimitiveType}
\newcommand{\AK}{\AgdaKeyword}
\newcommand{\AM}{\AgdaModule}
\newcommand{\AS}{\AgdaSymbol}
\newcommand{\AStr}{\AgdaString}
\newcommand{\AN}{\AgdaNumber}
\newcommand{\AD}{\AgdaDatatype}
\newcommand{\AF}{\AgdaFunction}
\newcommand{\AR}{\AgdaRecord}
\newcommand{\ARF}{\AgdaField}
\newcommand{\AB}{\AgdaBound}
\newcommand{\AIC}{\AgdaInductiveConstructor}
\newcommand{\AC}{\AgdaComment}

\newcommand{\Set}{\mathbf{Set}}

\strathsetidentity{Department of}{Computer \& Information Sciences}

\title{Generic Level Polymorphic N-ary Functions}
\author{Guillaume ALLAIS}
\institute{SPLS @ LFCS}
\date[Jun 17]{June 17, 2019}

\begin{document}

\begin{frame}[t]
\maketitle
\end{frame}

\begin{frame}
\tableofcontents
\end{frame}

\section{State Of the Art}

\subsection{N-ary Combinators... for N up to 2}

\begin{frame}{Propositional Equality}
  \ExecuteMetaData[StateOfTheArt.tex]{equality}
  \ExecuteMetaData[StateOfTheArt.tex]{cong}
  \ExecuteMetaData[StateOfTheArt.tex]{subst}
\end{frame}

\begin{frame}{Binary Versions}
  \ExecuteMetaData[StateOfTheArt.tex]{cong2}
  \ExecuteMetaData[StateOfTheArt.tex]{subst2}
\end{frame}

\begin{frame}{Wish: N-ary Versions}
  \hspace*{\mathindent}\begin{tabular}{@{}l@{~}l}
    \AF{congₙ} : & (\AB{f} : \AB{A₁} → ⋯ → \AB{Aₙ} → \AB{B}) →\\
                 & \AB{a₁} \AD{≡} \AB{b₁} → ⋯ → \AB{aₙ} \AD{≡} \AB{bₙ} → \\
                 & \AB{f} \AB{a₁} ⋯ \AB{aₙ} \AD{≡} \AB{f} \AB{b₁} ⋯ \AB{bₙ}
  \end{tabular}

  \medskip

  \hspace*{\mathindent}\begin{tabular}{@{}l@{~}l}
    \AF{substₙ} : & (\AB{R} : \AB{A₁} → ⋯ → \AB{Aₙ} → \AF{Set} \AB{r}) →\\
                  & \AB{a₁} \AD{≡} \AB{b₁} → ⋯ → \AB{aₙ} \AD{≡} \AB{bₙ} → \\
                  & \AB{R} \AB{a₁} ⋯ \AB{aₙ} → \AB{R} \AB{b₁} ⋯ \AB{bₙ}
  \end{tabular}
\end{frame}

\subsection{Working with Indexed Families}

\begin{frame}{List}
  \ExecuteMetaData[StateOfTheArt.tex]{list}
  \ExecuteMetaData[StateOfTheArt.tex]{all}
\end{frame}

\begin{frame}{Quantifiers}
  \ExecuteMetaData[StateOfTheArt.tex]{universal}
  \ExecuteMetaData[StateOfTheArt.tex]{iuniversal}
  \ExecuteMetaData[StateOfTheArt.tex]{replicate}
\end{frame}

\begin{frame}{Lifting of Type Constructors}
\ExecuteMetaData[StateOfTheArt.tex]{implies}
\ExecuteMetaData[StateOfTheArt.tex]{ap}
\end{frame}

\begin{frame}{Adjustments To The Ambient Index}
  \ExecuteMetaData[StateOfTheArt.tex]{update}
  \ExecuteMetaData[StateOfTheArt.tex]{join}
\end{frame}

\section{Requirements}

\begin{frame}{Wishes}
\end{frame}

\section{Getting Acquainted With the Unifier}

\begin{frame}{Unification}

  \begin{itemize}
    \item Use case\medskip

    Mechanical process to reconstruct missing values:
    \begin{itemize}
      \item Implicit arguments
      \item Boring details the programmer left out
    \end{itemize}

    Principled: the generated solutions (if any) are unique.

    \bigskip

    \item Unification Problems: {\AB{lhs} ≈ \AB{rhs}}\medskip

    \begin{itemize}
      \item \AB{?a} stands for a metavariable
      \item {\AB{e}\,[\AB{?a₁}, ⋯ ,\AB{?aₙ}]} for expression \AB{e} mentioning \AB{?a₁} to \AB{?aₙ}
      \item {\AB{c} \AB{e₁} ⋯ \AB{eₙ} for a} constructor \AB{c} applied to \AB{n} expressions
    \end{itemize}

  \end{itemize}
\end{frame}

\begin{frame}{Unification Tests}
  Agda does unification all the time.\medskip

  It is easy for us to ask Agda to solve unification problems

  \begin{itemize}
    \item Leave out values to create metavariables
    \item State that two expressions are equal to start a unification problem
  \end{itemize}

  \bigskip
  For instance, {(\AB{?A} → \AB{?B}) ≈ (\AD{ℕ} → \AD{ℕ})}:

  \ExecuteMetaData[Unifier.tex]{unifproblem}

\end{frame}

\begin{frame}{Instantiation}
  \begin{alertblock}{Problem: {\AB{?a} ≈ \AB{e}\,[\AB{?a₁} ⋯ \AB{?aₙ}]}}
    Unifying a meta-variable with an expression which does not mention it.
  \end{alertblock}

  \bigskip
  \begin{enumerate}
    \item Instantiate \AB{?a} to {\AB{e}\,[\AB{?a₁} ⋯ \AB{?aₙ}]}
    \item Discard the problem
  \end{enumerate}

  \bigskip Example:
  \ExecuteMetaData[Unifier.tex]{instantiation}
\end{frame}

\begin{frame}{Constructor Headed Problems}

  \begin{alertblock}{Problem: {\AB{c} \AB{e₁} ⋯ \AB{eₘ} ≈ \AB{d} \AB{f₁} ⋯ \AB{fₙ}}}
    Unifying two constructor-headed expressions.
  \end{alertblock}

  \bigskip
  \begin{enumerate}
    \item Make sure the constructors \AB{c} and \AB{d} are equal
    \item This means \AB{m} equals \AB{n}
    \item Replace problem with subproblems {(\AB{e₁} ≈ \AB{f₁}) ⋯ (\AB{eₘ} ≈ \AB{fₙ})}
  \end{enumerate}

  \bigskip Example:
  \ExecuteMetaData[Unifier.tex]{unifconstr}
\end{frame}

\begin{frame}{Avoid Computations... Unless (Part I)}

  Avoid generating unification problems involving recursive functions.

  \begin{minipage}{0.45\textwidth}
    \ExecuteMetaData[Unifier.tex]{nary}
  \end{minipage}\begin{minipage}{0.45\textwidth}
    \ExecuteMetaData[Unifier.tex]{unsolved}
  \end{minipage}
  \bigskip
  \onslide<2>{
    Unless the recursion goes away in the cases you are interested in.

    \begin{minipage}{0.45\textwidth}
      \ExecuteMetaData[Unifier.tex]{normalised0}
    \end{minipage}\begin{minipage}{0.45\textwidth}
      \ExecuteMetaData[Unifier.tex]{normalised1}
    \end{minipage}
  }
\end{frame}

\begin{frame}{Avoid Computations... Unless (Part II)}

  Avoid generating unification problems involving recursive functions.

  \begin{minipage}{0.45\textwidth}
    \ExecuteMetaData[Unifier.tex]{nary}
  \end{minipage}\begin{minipage}{0.45\textwidth}
    \ExecuteMetaData[Unifier.tex]{notinverted}
  \end{minipage}
  \bigskip
  \onslide<2>{
    Unless the recursion is trivially invertible.

    \begin{minipage}{0.45\textwidth}
      \ExecuteMetaData[Unifier.tex]{inverted0}
    \end{minipage}\begin{minipage}{0.45\textwidth}
      \ExecuteMetaData[Unifier.tex]{inverted}
    \end{minipage}
  }
\end{frame}

\section{Generic Level Polymorphic N-ary Functions}

\begin{frame}{Design Constraints}

  We want to
  \begin{itemize}
    \item Define representation of \AB{n}-ary functions
    \item Give it a semantics (here called \AF{⟦\_⟧})
  \end{itemize}

  \bigskip
  Such that when faced with constraints involving concrete types,
  Agda can easily reconstruct the representation.

  \bigskip
  Example: recover \AB{?r} from {\AF{⟦} \AB{?r} \AF{⟧} ≈ (\AD{ℕ} → \AF{Set})}
\end{frame}

\subsection{Unification-Friendly Representation}

\begin{frame}{Representation}
  \begin{minipage}{0.5\textwidth}
    \ExecuteMetaData[N-ary.tex]{levels}
  \end{minipage}\begin{minipage}{0.4\textwidth}
    \onslide<2->{\ExecuteMetaData[N-ary.tex]{tolevel}}
  \end{minipage}
  \onslide<3->{\ExecuteMetaData[N-ary.tex]{sets}}
  \onslide<4>{\ExecuteMetaData[N-ary.tex]{arrows}}
\end{frame}

\subsection{N-ary Combinators}

\subsection{Going Further}


\begin{frame}
\end{frame}

\end{document}
